\section{Workflow}

\subsection{GitHub}
\label{sub:GitHub}

To coordinate and control our coding, we have used GitHub. GitHub is version control software, which allows its users to have an overview of the development of a project. GitHub has many features and is mostly used by software engineers to software development. For us, GitHub was particularly useful because it forces you to document all the changes you make to the project. In this way, it was much easier for us two to work remotely and still don't lose of sight the work our partner had so far done. The git system is made to be used from the command line (Terminal for macOS or Command Line for Windows), but GitHub offers a very intuitive desktop application, which makes the life of not-so-advanced users much easier.

Alongside with the documentation function, the git system offers the possibility reload previous versions of the project and undo changes that turn out to be undesired. GitHub displays a timeline in which the user can follow the development of the project from its beginning until its current state. This allows the user togo back in time and recover old versions of project or only parts of it. This was particularly useful for us. It happened more than once that one of us two has mistakenly changed the code and Dynare could no longer simulate the model. Instead of spending hours debugging and trying to find the error, we simply opted to go back a previous version and redo the necessary steps.

\subsection{Syntax Highlighting}
\label{sub:Syntax Highlighting}

Another tip worth mentioning is to use syntax highlighting while coding. It is not only much more comfortable to read a code whose parts have different colors according to its function, but the task of spotting errors becomes much easier when the code is highlighted. Even though we could not find a specific syntax highlighting package for Dynare, it turns out that Java's highlighting works well with Dynare. For example, \% is used to comment the code and that single feature makes worth trying it.
