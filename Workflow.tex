\section{Workflow}

\subsection*{GitHub}
\label{sub:GitHub}

To coordinate and control our coding, we have used GitHub.\footnote{https://github.com} GitHub is version control software based on the git system. It allows its users to keep track of all the changes made to a specific project and provides an overview of the its development. GitHub has a variety of features and it is mostly used by software engineers who work simultaneously with many other people in big projects. GitHub was particularly useful to us because it forced us to document all the changes we made to the project. This way it was much easier for us two to work remotely without losing track of what each other had already done.
\par
\bigskip
Alongside with the documentation function, the git system offers the possibility to reload previous versions of the project and undo changes that turn out to be undesired. GitHub displays a timeline in which the user can follow the development of the project from its beginning until its current state. This allows the user to go back in time and recover old versions of project or only parts of it. This was particularly useful for us too. It happened more than once that one of us two has mistakenly changed the code and Dynare could no longer simulate the model. Instead of spending hours debugging and trying to find the error, we simply opted to go back to a previous version and start again from there.
\par
\bigskip
Although the git system is primarily designed to be used from the command line (Terminal for macOS or Command Line for Windows), GitHub offers a very user-friendly desktop application, which makes possible for not-so-advanced user to profit from its features. GitHub is free as long as you are willing to make your project public. The code for this project, for example, is available online.\footnote{https://github.com/denismaciel/fiscalfreelunch}


\subsection*{Syntax Highlighting}
\label{sub:Syntax Highlighting}

Another tip worth mentioning is to use syntax highlighting while coding. It is not only much more comfortable to read a code whose parts have different colors according to its function, but the task of spotting errors becomes much easier when the code is highlighted. Even though we could not find a specific syntax highlighting package for Dynare, it turns out that Java highlighting works well with it. For example, \% is used to comment the code and the single feature of being able to quickly tell the difference between actual code and comments makes it worth trying.
